%%%%%%%%%%%%%%%%%%%%%%%%%%%%%%%%% LAB-5 %%%%%%%%%%%%%%%%%%%%%%%%%%%%%%%%%%
%>>>>>>>>>>>>>>>>>>>>>>>>>> ПЕРЕМЕННЫЕ >>>>>>>>>>>>>>>>>>>>>>>>>>>>>>>>>>>
%>>>>> Информация о кафедре
%\newcommand{\year}{2021 г.}  % Год устанавливается автоматически
\newcommand{\city}{Санкт-Петербург}  %  Футер, нижний колонтитул на титульном листе
\newcommand{\university}{Национальный исследовательский университет ИТМО}  % первая строка
\newcommand{\department}{Факультет программной инженерии и компьютерной техники}  % Вторая строка
\newcommand{\major}{Направление программная инженерия}  % Треьтя строка
% Пусть будет. Проще закоментить лишнее.
\newcommand{\education}{Образовательная программа системное и прикладное программное обеспечение}  % четвертая строка
\newcommand{\specialization}{Специализация системное программное обеспечение}  % пятая строка

%<<<<< Информация о кафедре

%>>>>> Назание работы
\newcommand{\reporttype}{ОТЧЕТ ПО ДОМАШНЕЙ РАБОТЕ} % тип работы, (главный заголовок титульного листа)
\newcommand{\lab}{Домашняя работа}          % вид работы
\newcommand{\labnumber}{№ 3}                    % порядковый номер работы
\newcommand{\subject}{Разработка компиляторов}         % учебный предмет
\newcommand{\labtheme}{Конструирование LL(1)  анализатора для КС-грамматики}            % Тема лабораторной работы
\newcommand{\variant}{№ 9}                % номер варианта работы

\newcommand{\student}{Тюрин Иван Николаевич}    % определение ФИО студента
\newcommand{\studygroup}{P33102}                 % определение учебной группы 
\newcommand{\teacher}{% принимающий
  Лаздин А. В.% ФИО лектора
}
%<<<<<<<<<<<<<<<<<<<<<<<<<< ПЕРЕМЕННЫЕ <<<<<<<<<<<<<<<<<<<<<<<<<<<<<<<<<<<


%>>>>>>>>>>>>>>>>>>>>>> ПРЕАМБУЛА >>>>>>>>>>>>>>>>>>>>>>>>>
\include{preamble}
%<<<<<<<<<<<<<<<<<<<<<< ПРЕАМБУЛА <<<<<<<<<<<<<<<<<<<<<<<<<



%%%%%%%%%%%%%%%%%%% СОДЕРЖИМОЕ ОТЧЕТА %%%%%%%%%%%%%%%%%%%%%
%>>>>>>>>>>>>>>> ''''''''''''''''''''''' >>>>>>>>>>>>>>>>>>
\begin{document}


%>>>>>>>>>>>>>>>> ОПРЕДЕЛЕНИЕ НАЗВАНИЙ >>>>>>>>>>>>>>>>>>>>
% Переоформление некоторых стандартных названий
%\renewcommand{\chaptername}{Лабораторная работа}
\renewcommand{\chaptername}{\lab\ \labnumber} % переименование глав
\def\contentsname{Содержание} % переименование оглавления
%<<<<<<<<<<<<<<<< ОПРЕДЕЛЕНИЕ НАЗВАНИЙ <<<<<<<<<<<<<<<<<<<<
% \setlength{\itemsep}{0pt} % установка расстояния между строчками в списках можно использовать локально внутри списка списке
% \setlength{\parskip}{0pt} % 
% \setlength{\parsep}{0pt}  % 

%>>>>>>>>>>>>>>>>> ТИТУЛЬНАЯ СТРАНИЦА >>>>>>>>>>>>>>>>>>>>>
\include{titlepage}
%<<<<<<<<<<<<<<<<< ТИТУЛЬНАЯ СТРАНИЦА <<<<<<<<<<<<<<<<<<<<<


%>>>>>>>>>>>>>>>>>>>>> СОДЕРЖАНИЕ >>>>>>>>>>>>>>>>>>>>>>>>>
% Содержание
\tableofcontents
%<<<<<<<<<<<<<<<<<<<<< СОДЕРЖАНИЕ <<<<<<<<<<<<<<<<<<<<<<<<<


%%%%%%%%%%%%%%%%%%%%%%% КОД РАБОТЫ %%%%%%%%%%%%%%%%%%%%%%%%
%>>>>>>>>>>>>>>>>>>>'''''''''''''''''>>>>>>>>>>>>>>>>>>>>>
\newpage
\Chapter{\lab\ \labnumber}{\labtheme}{}

\Section{Задание варианта \variant}
 Исходный вариант:

\begin{align*}
S & \to ABCC\\
C & \to cccA  \;|\;  ccBB   \;|\;  cC  \;|\;  c\\
B & \to BBb  \;|\;  BBa  \;|\;  b\\
A & \to aAa  \;|\;  c \\
\end{align*}

Упрощение (убраны все подряд идущие нетерминалы, которые раскрываются в
бесконечные последовательности):

\begin{align*}
S & \to ABC\\
C & \to cccA  \;|\;  ccB   \;|\;  cC  \;|\;  c\\
B & \to Bb  \;|\;  Ba  \;|\;  b\\
A & \to aAa  \;|\;  c \\
\end{align*}

Факторизация C:

\begin{align*}
C & \to c c c A  \;|\;  c c B  \;|\;  c C  \;|\;  c\\
\end{align*}

\begin{align*}
C & \to c C_1\\
C_1 & \to c c A  \;|\;  c B  \;|\;  c C_1  \;|\;  \eps\\
\end{align*}

\begin{align*}
C & \to c C_1\\
C_1 & \to c C_2  \;|\;  \eps\\
C_2 & \to c A  \;|\;  B  \;|\;  C_1\\
\end{align*}

\begin{align*}
C & \to c C_1\\
C_1 & \to c C_2  \;|\;  \eps\\
C_2 & \to c A  \;|\;  B  \;|\;  c C_2  \;|\;  \eps\\
\end{align*}

\begin{align*}
C & \to c C_1\\
C_1 & \to c C_2  \;|\;  \eps\\
C_2 & \to c C_3  \;|\;  B  \;|\;  \eps\\
C_3 & \to A  \;|\;  C_2\\
\end{align*}

\begin{align*}
C & \to c C_1\\
C_1 & \to c C_2  \;|\;  \eps\\
C_2 & \to c C_3  \;|\;  B  \;|\;  \eps\\
C_3 & \to A  \;|\;  c C_3  \;|\;  B  \;|\;  \eps\\
\end{align*}

\begin{align*}
C & \to c C_1\\
C_1 & \to c C_2  \;|\;  \eps\\
C_2 & \to c C_3  \;|\;  B  \;|\;  \eps\\
C_3 & \to c  \;|\;  a A a  \;|\;  c C_3  \;|\;  B  \;|\;  \eps\\
\end{align*}

\begin{align*}
C & \to c C_1\\
C_1 & \to c C_2  \;|\;  \eps\\
C_2 & \to c C_3  \;|\;  B  \;|\;  \eps\\
C_3 & \to a A a  \;|\;  c C_4  \;|\;  B  \;|\;  \eps\\
C_4 & \to C_3  \;|\;  \eps\\
\end{align*}

Видно, что $C_4 = C_3$, поэтому уберем это правило:

\begin{align*}
C & \to c C_1\\
C_1 & \to c C_2  \;|\;  \eps\\
C_2 & \to c C_3  \;|\;  B  \;|\;  \eps\\
C_3 & \to a A a  \;|\;  c C_3  \;|\;  B  \;|\;  \eps\\
\end{align*}

Факторизация B:

\begin{align*}
B & \to B b  \;|\;  B a  \;|\;  b
\end{align*}

\begin{align*}
B & \to B B_1  \;|\;  b\\
B_1 & \to b  \;|\;  a\\
\end{align*}

Устраняем левую рекурсию B:

\begin{align*}
B & \to b B_2\\
B_1 & \to b  \;|\;  a\\
B_2 & \to B_1 B_2  \;|\;  \eps\\
\end{align*}

Итоговая грамматика:

\begin{align*}
S & \to A B C\\
C & \to c C_1\\
C_1 & \to c C_2  \;|\;  \eps\\
C_2 & \to c C_3  \;|\;  B  \;|\;  \eps\\
C_3 & \to A  \;|\;  c C_3  \;|\;  B  \;|\;  \eps\\
B & \to b B_2\\
B_1 & \to b  \;|\;  a\\
B_2 & \to B_1 B_2  \;|\;  \eps\\
A & \to a A a  \;|\;  c \\
\end{align*}

\begin{table}[H]
    \centering
    \resizebox{\textwidth}{!}{
    \begin{tabular}{c|c|c|c|c|c|c}
\textbf{FIRST}	    &	\textbf{FOLLOW}	  &	\textbf{Name}	&	\textbf{c}      & \textbf{a}      &	\textbf{b}        &	\textbf{\$}	  \\\hline\hline
$\set{a,c}		     $ & $\set{\$}	     $ & $S		$         & $S \to A B C	$   & $S \to A B C	  $ & $	              $ & $	           $ \\\hline
$\set{c}		       $ & $\set{\$}	     $ & $C		$         & $C \to c C_1	$   & $		            $ & $	              $ & $	           $ \\\hline
$\set{c,\eps}	   $ & $\set{\$}	     $ & $C_1 $         & $C_1 \to c C_2$   & $		            $ & $	              $ & $C_1 \to \eps$ \\\hline
$\set{c,\eps,b}	 $ & $\set{\$}	     $ & $C_2 $         & $C_2 \to c C_3$   & $		            $ & $C_2 \to B	    $ & $C_2 \to \eps$ \\\hline
$\set{a,c,\eps,b} $ & $\set{\$}		   $ & $C_3 $         & $C_3 \to c C_3$   & $C_3 \to a A a  $ & $C_3 \to B	    $ & $C_3 \to \eps$ \\\hline
$\set{b}		       $ & $\set{c,\$}		 $ & $B	  $         & $		          $   & $		            $ & $B \to b B_2    $ & $	           $ \\\hline
$\set{b,a}		     $ & $\set{b,a,c,\$} $ & $B_1 $         & $		          $   & $B_1 \to a		  $ & $B_1 \to b	    $ & $	           $ \\\hline
$\set{b,a,\eps}	 $ & $\set{c,\$}		 $ & $B_2 $         & $B_2 \to		  $   & $B_2 \to B_1 B_2$ & $B_2 \to B_1 B_2$ & $B_2 \to \eps$ \\\hline
$\set{a,c}		     $ & $\set{b,a}		   $ & $A		$         & $A \to c		  $   & $A \to a A a		$ & $		            $ & $	           $ \\\hline
    \end{tabular}
    }
    \caption{Сконструированный СА}
    \label{tab:constracted-sa}
\end{table}

Для сконструированного СА разработаем программный код выполняющий аналогичную функцию, см. \ref{lst:sa-code}. Программный код составляет код на языке DOT для визуализации в среде Graphviz.

\lstinputlisting[language=python, caption=Python-код синтаксического анализатора, label=lst:sa-code]{res/parser.py}

Пример работы программы для строки \verb|acabbc| можно видеть на листинге \ref{lst:sa-output}. И получившийся граф можно видеть на рисунке \ref{fig:sa-graph}.


\begin{lstlisting}[caption=Пример вывода программы для строки acabbc, label=lst:sa-output]
digraph {
    0 [label = "S"];
    1 [label = "A"];
    0 -> 1;
    2 [label = "A"];
    1 -> 2;
    3 [label = "B"];
    0 -> 3;
    4 [label = "B2"];
    3 -> 4;
    5 [label = "B1"];
    4 -> 5;
    6 [label = "B2"];
    4 -> 6;
    7 [label = "C"];
    0 -> 7;
    8 [label = "C1"];
    7 -> 8;
}
\end{lstlisting}


\begin{figure}
    \centering
    \includegraphics[width=0.5\linewidth]{res/graph.png}
    \caption{Получившийся граф разбора для строки acabbc}
    \label{fig:sa-graph}
\end{figure}

\Section{Вывод}

Изучили принципы построения синтаксических анализаторов, способы устранения левой рекурсии и выполнения левой факторизации. Построили синтаксический анализатор для  грамматики выданной по варианту и реализовали его поведение в программном коде.  



%<<<<<<<<<<<<<<<<<<<<<< КОД РАБОТЫ <<<<<<<<<<<<<<<<<<<<<<<<

\end{document}
%<<<<<<<<<<<<<<<< ,,,,,,,,,,,,,,,,,,,,,,, <<<<<<<<<<<<<<<<<
%<<<<<<<<<<<<<<<<<<< СОДЕРЖИМОЕ ОТЧЕТА <<<<<<<<<<<<<<<<<<<<
